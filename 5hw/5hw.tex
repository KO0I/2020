\documentclass{hw}
\usepackage{amsmath,listings,xcolor,graphicx,circuitikz}
\title{Homework 5}
\course{ECEN 2020}
\author{Patrick Harrington\\Maxim Allen}
\renewcommand\thesubsection{(\alph{subsection})}
\lstset{ 
  language=C,
  basicstyle=\fontfamily{pcr}\selectfont\footnotesize\color{black},
  keywordstyle=\color{black}\bfseries, % style for keywords
  numberstyle=\tiny\color{gray},
  rulecolor=\color{black},
  numbers=left, numberstyle=\tiny, 
  frame=single,
  tabsize=2
}
\begin{document}
\maketitle
\section{Analog-to-Digital Conversion Techniques}
%%%%%%%%%%%%%%%%%%%%%%%%%%%%%%%%%%%%%%%%%%%%%%%%%%%%%%%%%%%%%%%%%%%%
To convert an analog signal into something digital and readable, the
microcontroller samples the signal at a rate that is at least twice
the frequency of the signal. When deciding what values can be
represented from the incomming data, we can use $2^n - 1$ bits where n
decides the total number of different 'levels' we can represent. if
we had 1 bit total, we can represent 2 different levels, on or off.
when we have $n = 8$, we get 127 levels that we can represent from the
digital signal. the more bits we assign the better. 
Analog to digital conversion is a lot like approximating an integral
in calculus, you can take a sample of what the data is at a given
change in time and calculate the value at that point.

\section{Voltage Division}
%%%%%%%%%%%%%%%%%%%%%%%%%%%%%%%%%%%%%%%%%%%%%%%%%%%%%%%%%%%%%%%%%%%%
Using \texttt{analogRead()}, the voltage across a pair of resistors
$R_1=22 [k \Omega]$ and $R_2 =270 [\Omega]$. From
Equation~(\ref{eq:vdr}), we have calculated the expected voltage
from the circuit shown in Figure~\ref{fig:vdr}.
\begin{equation}
  V_o = \frac{R_2}{R_1 + R_2} \cdot 5[V] = 60.6[mV]
  \label{eq:vdr}
\end{equation}
We also tested again with two equal resistors found an expected drop
of $2.5[V]$, since two resistors in parallel each drop half of the
supplied voltage. We also noticed that the Arduino was less accurate
with our original resistor values because the voltage dropped was so
significant, that the tiny detected voltage was subject to considerable
noise.
\begin{figure}[ht!]
  \centering
  \begin{circuitikz}\draw
    (0,4) node[anchor=east]{$5 [V]$}
    to[R,l=$R_1$,o-*](3,4)
    to[R,l=$R_2$,*-](3,1)node[ground]{}
    (3,4)to[short,*-o](6,4) node[anchor=west]{$V_o$}
    ;\end{circuitikz}
  \caption{Voltage Divider Circuit}
  \label{fig:vdr}
\end{figure}

The code used to measure this quantity is attached below:

\lstinputlisting[language=C]{./harrington_allen_HW5/analogread.ino}

\section{Advantages of Working at the Register Level}
%%%%%%%%%%%%%%%%%%%%%%%%%%%%%%%%%%%%%%%%%%%%%%%%%%%%%%%%%%%%%%%%%%%%
This question asked about manipulating ports on a register level
rather than writing their states with functions. This is beneficial
because libraries can take up a lot of memory space. especially when
you call them multiple times. the following code snippet calls digital write
4 times:
\begin{lstlisting}
  digitalWrite(0, HIGH);
  digitalWrite(1, HIGH);
  digitalWrite(2, HIGH);
  digitalWrite(3, HIGH);
\end{lstlisting}

This takes up a lot more memory and processing power than \texttt{PORTA |=
0x0F;} this method requires no extra space for libraries and is
very minimalistic in it processing requirements.

\section{Time Elapsed since Program Execution}

As directed, the following code includes use of the serial monitor
and the Arduino libraries.

\lstinputlisting[language=C]{./harrington_allen_HW5/runtime.ino}

\section{Double Blink}
%%%%%%%%%%%%%%%%%%%%%%%%%%%%%%%%%%%%%%%%%%%%%%%%%%%%%%%%%%%%%%%%%%%%

Functional code was developed, but this depended upon the libraries.
Also included in the homework submission is a copy of an attempt to
use ISRs to generate seperate clock signals, but one of the timers
(\texttt{TCCR2A}) was problematic because it was an 8-bit timer.

\lstinputlisting[language=C]{./harrington_allen_HW5/DoubleBlink.ino}
\section{Signal Processing}
%%%%%%%%%%%%%%%%%%%%%%%%%%%%%%%%%%%%%%%%%%%%%%%%%%%%%%%%%%%%%%%%%%%%
\subsection{}The period of this wave is $T=6 ms$, so it is oscillating
at $f\approx 166 [Hz]$. thus, we need to sample it at at least 332 hz to get a
good reading.

\subsection{} If you only checked this signal once per second, you
are skipping 165 cycles of possible data change. The resultant
sampled signal would be unrecognizable!

\subsection{} Aliasing in signal processing is when an analog signal
is converted to digital and reconstructed using the digital values.
during this reconstruction, data points are rounded off or grouped
together making the reconstructed signal rough in comparison to what
the actual signal is. Aliasing also happens when the signal isnt
sampled fast enough so some data points are completely neglected. An
aliased signal may appear completely different than the actual
signal being sampled.

\subsection{}Using the same logic from Equation~(\ref{eq:vdr}),
$10[k\Omega]$ is the required $R_2$ value. When voltage dividing
into half the voltage you have, both resistors must be equal.

\subsection{}Protecting the arduino from high voltages involves a
voltage divider. Protecting the arduino from large currents is a
little bit tougher, but can be done with just a current divider. The
equation for current division is shown in Equation~(\ref{eq:cdr}).

\begin{equation}
  I_o = \frac{G_x}{G_{total}}\cdot I_t
  \label{eq:cdr}
\end{equation}

Where $G_x=\frac{1}{R_x}$ and $G_{total}$ for $n$ resistors is
defined as:

\begin{align*}
  G_{total} = \sum_1^n \frac{1}{R_i}
\end{align*}

\end{document}
