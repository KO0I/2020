\documentclass{hw}
\usepackage{amssymb,listings,xcolor}
\title{Homework 1}
\course{ECEN 2020}
\author{Patrick Harrington}
\renewcommand\thesubsection{(\alph{subsection})}
\lstset{ 
  language=C,
  basicstyle=\fontfamily{pcr}\selectfont\footnotesize\color{black},
  keywordstyle=\color{black}\bfseries, % style for keywords
  numberstyle=\tiny\color{gray},
  rulecolor=\color{black},
  numbers=left, numberstyle=\tiny, 
  frame=single,
  tabsize=2
}
\begin{document}
\maketitle
\section{Variable Types}
\subsection{Size of integer on computer and Arduino}
The size of integers on my computer (in bytes) is 4 bytes (found by using C's
sizeof(int) function). On the Arduino, the integer size is smaller at 2 bytes.
\subsection{What is a type cast? Why is it useful?}
A type cast is an explicit re-definition of a data value's type. Type casting
would allow, for instance, an integer to be converted into a double. The data
itself remains unchanged, but the format the data represents changes. This has
many uses in arithmetic operations. One example is truncating doubles or
floats.
\subsection{Simply explain what preprocessor directives are and what they can
be used for}
A preprocessor directive is used most often to include files or to set up
global variables. In C, preprocessor directives are initiated with a
``\texttt{\#}''
symbol.
\section{Explain the following section of code}
\begin{lstlisting}
int* x;
int* y;
x = malloc(sizeof(int));
y = malloc(sizeof(int));
*x = 1;
*y = 2;
x=y;
printf(``%d, %d\n'', x ,y);
free(x);
free(y);
\end{lstlisting}

First, x and y are initialized as pointers to \texttt{ints}. In lines 3 and 4,
x and y are reserving data that is exactly an integer's size in bytes. Running
this code on my laptop would allocate 4 bytes off the heap. In lines 5 and 6,
the pointers \texttt{x} and \texttt{y} are each assigned values in memory. In
line 7, point \texttt{x} is then set to point to \texttt{y}. When the program
is ran, two numbers are printed because of line 8's \texttt{printf} call. The
numbers are equal, and represent the value that \texttt{y} points two
with a memory address at 2. This can be seen from the output of
\texttt{printf(``\%d, \%d\\n'', *x, *y);} if ran before and after the
assignment at line 7.

\section{Loops}
\begin{lstlisting}
void setup() {}
void loop() { my_function();}
void my_function()
{
  static int MyVar1 = 0;
  int MyVar2 = 0;
  MyVar1++;
  MyVar2++;
}
\end{lstlisting}
\subsection{What are the values of \texttt{MyVar1} and \texttt{MyVar2} at the end
of \texttt{my\_function} for the first five calls to \texttt{my\_function}?}
At the end of five calls to \texttt{my\_function}, the variable
\texttt{MyVar1}=5, and \texttt{MyVar2}=0. If both variables are set to be
\texttt{static}, they increment at the same time.
\subsection{What would the values be if these variables were the same type, but
global?}
If both variables are global, that is, if they are set before \texttt{void
setup()}, the output is quite different. Both \texttt{MyVar1} and
\texttt{MyVar2} increment up at the same time. At the end, both variales are
equal to 5.

\section{Given the following variables}
\begin{lstlisting}
const boolean bPenguin  = true;
const boolean bFrog     = false;
const int iTurtle       = 0x19;
const int iRabbit       = B00001111;
const int iHampster     = 0;
\end{lstlisting}
\newpage
\subsection{What would the result of the following statements be?}
\begin{lstlisting}
iTurtle   &   iRabbit;
iTurtle   &&  iRabbit;
bPenguin  &&  iRabbit;
iHampster &   iTurtle;
iTurtle   ||  iRabbit;
iTurtle   |   iRabbit;
iTurtle   |   bFrog;
iHampster ||  bPenguin;
\end{lstlisting}

\section{Explain the result of the following code}

\begin{lstlisting}
union data {
  unsigned char temp;
  unsigned int  time;
};
int main(){
  union data myData;
  myData.time = 0xFCAB;
  myData.temp = 0x0;

  printf(``Address of myData.time: '');
  printf(``%x\n'', (int)&myData.time);
  printf(``Address of myData.temp: '');
  printf(``%x\n'', (int)&myData.temp);
  printf(``\n'');
  printf(``Value of myData.time: '');
  printf(``%x\n'', myData.time);
  printf(``Value of myData.temp: '');
  printf(``%x\n'', myData.temp);
  return 0;
}
\end{lstlisting}

Running the code on an Arduino gave the following output:
\end{document}

