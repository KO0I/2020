\documentclass{hw}
\usepackage{amssymb,listings,xcolor,graphicx}
\title{Homework 2}
\course{ECEN 2020}
\author{Patrick Harrington}
\renewcommand\thesubsection{(\alph{subsection})}
\lstset{ 
  language=C,
  basicstyle=\fontfamily{pcr}\selectfont\footnotesize\color{black},
  keywordstyle=\color{black}\bfseries, % style for keywords
  numberstyle=\tiny\color{gray},
  rulecolor=\color{black},
  numbers=left, numberstyle=\tiny, 
  frame=single,
  tabsize=2
}
\begin{document}
\maketitle

\section{AVR Status Register (SREG)}
  The AVR status register contains information about the state of the
  microcontroller's processor - specificaly, the most recently executed arithmetic
  instruction. The memory-mapped address of the SREG starts at
  \texttt{0x3F}, and continues to \texttt{0x5f}. The default value is 
0. Bit 1, also called Z, is the \textbf{Zero Flag}. It indicates ``false'' for
arithmetic and logic operations, and is useful for branching conditional
statements.

\section{Checking AVR Status Register SREG}
See attached files.

The output was:

\texttt{Bit 0 Status: On}

\texttt{Bit 1 Status: On}

\texttt{Bit 2 Status: On}

\texttt{Bit 3 Status: On}

\texttt{Bit 4 Status: On}

\texttt{Bit 5 Status: On}

\texttt{Bit 6 Status: Off}

\texttt{Bit 7 Status: Off}

%\lstinputlisting[language=C++]{file}

\section{Low-Level Blink}
See attached files
\section{Lowest Frequency for Low-Level Blink}
The slowest I could get the LED to blink was 14.4Hz. The pulse width was
found to be about 37 ms. This was using unmodified code from problem 3; as I
did not use nested \texttt{for} loops.
\section{Memory Addresses, Variables and Definitions}
\subsection{Memory Range}
  If a variable is defined as \texttt{int Count\_Rotations=0}, it is likely
  stored in the \texttt{0x0100 - 0x08F7} range, assuming a total of 2KBytes of
  memory. This address range is the range of addresses that exist for the
  Atmega SRAM, minus half of the width of an integer, which is 2 Bytes long.
  \subsection{Array Definition}

  There is a problem with the following array definition: 
  
  \texttt{volatile int
    my\_array[2048]=\{0\};}, 

    Namely, it is with the value it is initialized with; it matches the
    size of all SRAM on the Atmega328! It would cause an immediate overflow.

\subsection{Two Variables}
Assuming that the first variable is at address \texttt{0x0120}:

\texttt{double my\_number = 5;}

\texttt{char my\_letter = 'b';}

The size of a \texttt{double} is 4 Bytes, and the size of a \texttt{char} is
only one Byte. As a result, the address of \texttt{b} is most likely
\texttt{0x0140}.

\section{Harvard vs. Von Neumann Architectures}
The key difference between the Von Neumann and Harcard architectures is in the
nature of the memory bus; with Von Neumann, the bus between program memory and
data memory is shared, leading to a bottleneck as data and instructions
have to be scheduled. The Harvard architecture has
seperate buses for data and program memory, allowing data and instructions
to be moving around at the same time. So while although the Von Neumann
architecture is simpler, it performs poorly compared to the Harvard
architecture. Another difference is in the use of caches. A Von Neumann
processor would have a single cache storing everything, whereas a Harvard
processor would need a cache for every bus to be efficient.

\end{document}

